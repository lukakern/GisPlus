\documentclass[10pt, a4paper]{article}

\usepackage{graphicx} 
\usepackage{natbib}
\bibpunct{(}{)}{;}{a}{}{,}  % to adjust punctuation in references
\usepackage[utf8]{inputenc}
\usepackage[margin=10pt, font=small, labelfont=bf]{caption} 
\usepackage[margin=1in]{geometry}
\usepackage{color}
\usepackage{xcolor}
\usepackage[]{hyperref}
\definecolor{darkblue}{rgb}{0,0,.5}
\hypersetup{colorlinks=true, breaklinks=true, linkcolor=darkblue, menucolor=darkblue, urlcolor=darkblue, citecolor=darkblue}
\usepackage{multicol}              
\usepackage{multirow}
\usepackage{booktabs}  
\usepackage{enumerate}
\usepackage{subcaption}
\usepackage{eurosym}
\usepackage{color}
\usepackage{siunitx}
\usepackage{lineno} % for line numbers 
\usepackage{setspace}
%\usepackage{float}
\usepackage{listings}
\usepackage{xfrac}
\usepackage[T1]{fontenc}


%\definecolor{middlegray}{rgb}{0.5,0.5,0.5}
%\definecolor{lightgray}{rgb}{0.8,0.8,0.8}%
%\definecolor{orange}{rgb}{0.8,0.3,0.3}
%\definecolor{yac}{rgb}{0.6,0.6,0.1}

%\lstset{
%	basicstyle=\small\ttfamily,
%	keywordstyle=\bfseries\ttfamily\color{orange},
%	stringstyle=\color{blue}\ttfamily,
%	commentstyle=\color{teal}\ttfamily,
%	emph={square}, 
%	emphstyle=\color{blue}\texttt,
%	emph={[2]root,base},
%	emphstyle={[2]\color{yac}\texttt},
%%	flexiblecolumns=false,
	%tabsize=2,
	%xleftmargin=5pt
%}

% Default fixed font does not support bold face
\DeclareFixedFont{\ttb}{T1}{txtt}{bx}{n}{12} % for bold
\DeclareFixedFont{\ttm}{T1}{txtt}{m}{n}{12}  % for normal

% Custom colors
\usepackage{color}
\definecolor{deepblue}{rgb}{0,0,0.5}
\definecolor{deepred}{rgb}{0.6,0,0}
\definecolor{deepgreen}{rgb}{0,0.5,0}
% Python style for highlighting
\newcommand\pythonstyle{\lstset{
		language=Python,
		basicstyle=\ttm,
		otherkeywords={self},             % Add keywords here
		keywordstyle=\ttb\color{deepblue},
		emph={MyClass,__init__},          % Custom highlighting
		emphstyle=\ttb\color{deepred},    % Custom highlighting style
		stringstyle=\color{deepgreen},
		frame=tb,                         % Any extra options here
		showstringspaces=false            % 
}}

\setlength\parindent{0pt}

\begin{document}

\markboth{}
\\\noindent University of Freiburg \hspace{10cm}  July 6, 2018
\\Faculty of Environment and Natural Resources
\\Module: GIS Plus
\\Lecturer: João Paulo Pereira, Holger Weinacker
\\Authors: Luka Kern, Nele Stackelberg and Felix Rentschler
\\

\begin{center}
	\huge{Project 06: Rasterizer Function} \vspace{0.5cm}\\
	%\Large {Rasterizer}
\end{center}

%\maketitle
\
\onehalfspacing % larger vertical space between lines 
%\linenumbers



\begin{abstract}
	A rasterinzing tool creates a regular grid whithin the bounding box of a geometry collection and assigns a value to each cell depending on the presence of a geometry in that cell region. This information can either be a binary one (no-data / data) or based on the geometry attributes. For this Project, the first cell will create a randomized geometry list containing the input data. The attributes for each of the three geometries are stored in a separated attributes list.
\end{abstract}


\section{Aim of the Project}
The aim of this project was to create a function that generates a raster-file (.tif). This file displays the geometries from a shapfile, which was given as input to the function. The spatial resolution and name of the output-file can be defined as input to the function.
For the case that no shapefile is available to run the function, a code to generate random geometries is provided.


\section{Structure of the function}
\subsection{Load in the shapefile}
To load in the shapefile, the fiona-package is used. Every geometry-object of the shapefile is stored in a GeometryCollection from the shapely-package.

\subsection{Defining the bounding box}
We create a buffer around the actual shape of the shapefile which will look prettier in the resulting image. 

\newpage
% Python environment
\lstnewenvironment{python}[1][]
{
	\pythonstyle
	\lstset{#1}
}
{}


\begin{python}
class MyClass(Yourclass):
		def __init__(self, my, yours):
		bla = '5 1 2 3 4'
		print bla
\end{python}
\subsection{R Code}
\begin{lstlisting}[language=Python]
#=============================================================================
#                       Initial Values
#=============================================================================
def rasterizer(filepath,
							pixels=100,
						buffer=10,
outputname="output.tiff",
save=True,
preview=True):

\end{lstlisting}


\section{Issues}
\
\subsection{Solved}
\subsection{Unsolved}


\begin{figure}[!htbp]
	\centering
%	\includegraphics[width=16cm]{wildfire.png}
%	\caption{Wildfire event in a boreal forest}
\end{figure}




\end{document}